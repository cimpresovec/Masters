\documentclass[12pt,a4paper,twoside]{book}
\usepackage[utf8]{inputenc}
\usepackage{amsmath}
\usepackage{amsfonts}
\usepackage{amssymb}
\usepackage{graphicx}
\usepackage[inner=3.00cm, outer=2.50cm, top=3.00cm, bottom=3.00cm]{geometry}
\usepackage[slovene]{babel}
\usepackage{babelbib}
\usepackage{titlesec}
\usepackage{fancyhdr}
\usepackage{url}
\author{Luka Horvat}
\title{Dobre prakse pri razvoju računalniških iger}

%Paragraph indenting, title spacing and line spacing
\setlength{\parindent}{0pt}
\setlength{\parskip}{1em}
\renewcommand{\baselinestretch}{1.5}

% Remove big chapter text
\titleformat{\chapter}{\bfseries\LARGE}{\thechapter}{1em}{\LARGE\textbf}

%Header/footer
\pagestyle{fancy}
\fancyhf{}
\fancyfoot[LE,RO]{\thepage}
\renewcommand{\headrulewidth}{0pt}
\renewcommand{\footrulewidth}{0pt}

%Initial pages are in Roman numbering
\pagenumbering{Roman}
\begin{document}
	
%First page
\thispagestyle{empty} 
\begin{center}
{\large 
UNIVERZA V MARIBORU\\
FAKULTETA ZA ELEKTROTEHNIKO,\\
RAČUNALNIŠTVO IN INFORMATIKO\\
}

\vspace{\fill}
{\LARGE Luka Horvat}\\

\vspace{1cm}
\textsc{\textbf{\LARGE
		DOBRE PRAKSE PRI RAZVOJU RAČUNALNIŠKIH IGER\\}}

\vspace{1cm}
{\LARGE Magistrsko delo}

\vfill
{\Large Maribor, februar 2018}
\newpage
\end{center}

%Empty page
\ \thispagestyle{empty}
\newpage

%Second page
\thispagestyle{empty} 
\begin{center}	
\vspace*{\fill}
\textsc{\textbf{\LARGE
		Dobre prakse pri razvoju računalniških iger\\
	}}
{\large\textbf{Magistrsko delo\\}
	
}
\vspace{\fill}
\begin{tabbing}
\hspace*{4cm}\=\hspace*{3cm}\= \kill
Študent: \> Luka Horvat\\
Študijski program: \> Študijski program 2. stopnje\\
\>Računalništvo in informacijske tehnologije\\
Mentor: \> doc. dr. Matej Črepinšek
\end{tabbing}
\end{center}
\newpage

%Empty page
\ \thispagestyle{empty}
\newpage

%Sklep
\thispagestyle{empty}
Tukaj pride sklep o potrjeni temi.
\newpage

%Empty page
\ \thispagestyle{empty}
\newpage

%Povzetek v slovenskem jeziku
\chapter*{Dobre prakse pri razvoju računalniških iger}
\thispagestyle{fancy}
\setcounter{page}{1}
\textbf{Ključne besede:} beseda1

\textbf{UDK:} 123

\textbf{Povzetek}\newline
\textit{Povzetek do maksimalne dolžine 100 besed}
\cleardoublepage

%Povzetek v angleškem jeziku
\chapter*{Dobre prakse pri razvoju računalniških iger}
\thispagestyle{fancy}
\textbf{Key words:} word 1

\textbf{UDK:} 123

\textbf{Abstract:}\newline
\textit{Povzetek do maksimalne dolžine 100 besed}
\cleardoublepage

%Kazalo
\tableofcontents
\thispagestyle{fancy}

%Content
%Uvod
\chapter{Uvod}
%Page numbering and style changes here
\setcounter{page}{1}
\pagenumbering{arabic}
\thispagestyle{fancy}

Računalniške igre so ena izmed največjih panog v zabaviščni industriji. Skozi leta pa se njihov delež samo povečuje. Dandanes se je že skoraj vsak posameznik srečal z računalniškimi igrami, ali jih neposredno igra v svojem prostem času, ali pa so se posredno vključile v kulturo okoli posameznika. Računalniške igre in še posebej like iz njih velikokrat vidimo v drugih medijih kot so filmi, serije, reklame ter tiskano gradivo. Kdo pa si danes pod imenom Mario ne predstavlja vodovodarja v rdečem kombinezonu ter velikimi brki?

Računalniške igre privabljajo vedno več podjetij in individualnih razvijalcev, ki poskušajo zavzet svoj prostor v tej industriji. Samo izdelovanje računalniških iger pa uvrstimo pod razvoj programske opreme, ki je izredno kompleksen in zahteva potrebna znanja iz več različnih panog. Napredek tehnologije pa je že skoraj vsakemu posamezniku omogočil preprost vstop v ta proces. Dandanes je na voljo toliko različnih orodij, knjig, procesov in ustaljenih praks, ki so na voljo posamezniku, vendar se v oceanu podatkov hitro izgubijo. Tako smo si kot cilj tega magistrskega dela zadali poiskati, opisati in preizkusit dobre ter priporočene prakse pri razvoju računalniških iger. Zajeli smo celoten spekter procesa, od same začetne ideje do izdaje igre na trg. Večji del magistrskega dela smo posvetili bolj tehnološko usmerjenim procesom pri izdelavi računalniške igre, pri čemer smo prikazali različna orodja in prakse, ki so dandanes na voljo.

TODO Povzetek poglavij.

%Main content
\chapter{Pregled in opis glavnih korakov razvoja}
\thispagestyle{fancy}
\section{Opis problema}
\label{sec:opis_problema}
Razvoj računalniških iger je proces razvoja programske opreme. Obenem je kreativen proces v katerem sodeluje širok spekter ljudi iz različnih panog. Glavne vloge v tem procesu so \cite{rogers2014level}:
\begin{itemize}
	\item \textbf{Programer:} je odgovoren za tehnično implementacijo računalniške igre. Odvisno od velikosti projekta in končnega cilja uporablja obstoječe namenske pogone za razvoj računalniških iger (npr. Unity ali Unreal Engine) ali implementira \textit{"in-house"} pogon za to specifično igro. Tukaj pridejo v poštev različna znanja iz 2D in 3D grafike, fizike, umetne inteligence, uporabniških vmesnikov, računalniških mrež, ipd. Obenem je še potreben ves spekter računalniškega znanja, ki se navezuje na principe specifične za računalniške igre.
	\item \textbf{Umetnik:} je odgovoren za izdelavo vseh grafičnih gradnikov. Določeni umetniki izrisujejo samo konceptne slike in s oblikovalcem poskusijo definirat končni cilj igre, kako bo igra izgledala, kaki bodo glavni liki, ipd. Preostali umetniki pripravljajo gradnike, ki se nato uporabijo v sami računalniški igri. To so lahko grafični gradniki za menije, 3D modeli objektov, teksture za le-te modele, vizualni efekti, animacije, ipd.
	\item \textbf{Oblikovalec:} je odgovoren za koncipiranje in definiranje računalniške igre. Išče in definira ideje, ki bodo in so pomembne za igro. Pomembna vrlina je, da je zmožen te ideje dobro komunicirati preostali ekipi, da jo le-ti potem uresničijo. Določeni oblikovalci se lahko osredotočijo na specifične aspekte igre, kot so stopnje v igri in določeni sistemi (npr. sistem bojevanja).
	Specifična panoga oblikovalca je tudi pisatelj. On je odgovoren in napiše zgodbo za igro. Včasih je sama zgodba rdeča nit in je glavna gonilna sila za nastanek igre, drugič pa se zgodba komaj oblikuje skozi nastanek končnega produkta.
	\item \textbf{Skladatelj in oblikovalec zvoka:} je odgovoren za vse zvočne gradnike igre. Skladatelji pripravljajo glasbo v igri. Dandanes pri večjih projektih sodelujejo tukaj celotni orkestri. Oblikovalec zvoka pa je odgovoren za različne krajše zvoke, ki se prožijo ob določenih akcijah (npr. streli orožij).
	\item \textbf{Preizkuševalec:} testira produkt skozi vse faze razvoja in daje povratne informacije drugim razvijalcem za izboljšanje produkta. Zagotavlja kakovost igre, da bo ta izšla brez hujših napak.
\end{itemize}
Poleg naštetih vlog je še veliko drugih na področju izdajanja igre, vodenje ekipe, marketinga ipd. vendar so zgornje najbolj pomembne pri samem razvoju igre. Znanja, ki jih te vloge premorejo so pomembne za razvoj vsake igre. Če gre za večji projekt, potem je potrebnih več ljudi v vsaki vlogi. Dandanes pri velikih podjetjih sodeluje preko sto ljudi pri razvoju iger. Na trgu pa obstaja tudi velika scena individualnih razvijalcev, ki sami premorejo vsa potrebna znanja za razvoj in tako uresničijo razvoj svoje igre. 

V tem magistrskem delu smo se posvetili individualnemu razvoju računalniške igre, in prakse, ki temu pristopu najbolj ustrezajo. Kot posamezni razvijalec je potrebno uporabiti znanja iz različnih panog, zato smo s tem delom poskusili zaobjeti prakse, ki bi olajšale le ta proces. V delu smo našteli in opisali različna orodja in procese za vsak korak v razvoju. Glavni koraki razvoja so sledeči: koncipiranje in definiranje igre, planiranje in vodenje dela, tehnični razvoj igre (tukaj pride v poštev izdelava grafičnih in zvočnih gradnikov ter implementacija), oglaševanje in viri občinstva ter na konci izdaja igre.

\section{Koncipiranje igre}
Prvi korak razvoja igre je definirat, kaj hočemo ustvarit. Lahko, da ima nekdo svojo sanjsko idejo, ki jo hoče uresničiti. Drugim se nenadoma prikrade ideja, ki jo potem razvijejo v delujoč produkt. Velikokrat pa je potrebno vložiti delo, da samo definiramo originalno idejo igre. Studiji, katerim je razvoj iger glavni vir prihodka, se ne morejo zanašati na naključne ideje. Obenem je izvirna in dobro premišljena ideja pomembna, če hočemo kot individualna oseba razviti svojo igro, saj je trg danes zelo nasičen in je prvi vtis zelo pomemben. Spodaj smo našteli nekaj načinov, kako vzpodbuditi ustvarjalnost in poiskati novo idejo za igro \cite{rogers2014level}: 
\begin{itemize}
	\item \textbf{Igranje obstoječe (slabe) igre.} Igre konstantno gradijo na idejah in mehanikah obstoječih iger. Dodajajo se novi koncepti in pravila, ki iz starih idej zgradijo nove izkušnje. Za vsako komercialno igro se lahko našteje ducat drugih, ki so služile kot navdih in podlaga za njo. Tako je dober način za iskanje idej igranje že obstoječih. Še boljše je igranje slabih iger in razmislit, kaj bi spremenili, da bi izkušnjo izboljšali. Tako iteriramo na obstoječih rešitvah, da definiramo nekaj boljšega.
	\item \textbf{Branje tematike izven osebnega zanimanja.} Razširjanje osebnega obzorja znanja in izkušenj je vedno dober vir novih idej. Velikokrat je problem pri igrah, da so si preveč podobne. Dandanes se zelo pogosto dogaja, da vsi poskusijo kopirat idejo z nove uspešne igre. Zato je pomembno, da raziščemo tematike, ki jih ne poznamo najbolje in poskusimo tam najti nove ideje.
	\item \textbf{Udeleževanje konferenc in predavanj.} Konference so vedno polne različnih ljudi z novimi idejami. V takem okolju hitro najdemo inspiracijo, ki jo potem izkoristimo za razvoj nove ideje.
	\item \textbf{Brainstorming}. Uveljavljena praksa iskanja novih idej. Pomembno je, da zberemo ljudi iz različnih področij in se držimo glavnega pravila: poudarek je na količini idej, ki se ne obsojajo \cite{osborn1953applied}. Tako poskusimo raziskat čim več idej, ki so lahko prvotno napačne, na katerih nato gradimo vedno boljše rešitve.
\end{itemize}
Pri iskanju ideje je pomembno, da se zavedamo svojih omejitev. Če smo posameznik, ki hoče uresničiti svojo prvo igro, potem je priporočljivo, da začnemo z preprostimi idejami. Take igre lahko potem uresničimo in postopoma povečujemo obsežnost naslednjih iger. Veliko posameznikov namreč začne z preveč ambicioznimi idejami, ko pa je potrebno projekt dejansko uresničit pa pride do komplikacij in večinoma do zaključitve projekta.
\section{Potek razvoja igre}
Razvoj igre skoraj vedno poteka po določenih korakih, neodvisno od ideje. Glavni mejniki procesa so:
\begin{enumerate}
	\item \textbf{Projektni dokument}. Tukaj preide ideja iz glave na fizični medij. Proces je zelo odvisen od velikosti projekta in studia. Pri večjih projektih nastane dejanski dokument, v katerem se definira ideja, kake bodo mehanike igre, zgodba, liki, razporeditev sredstev, predviden čas izdelave, ipd. Dokument je velikokrat tudi opremljen z konceptnimi slikami, ki jih pripravijo umetniki. S tem dokumentom je potem najlaže komunicirat vizijo preostalim ljudem v ekipi. Pri manjših idejah in posameznikih je ta korak bolj direkten. Ustvarijo se kake skice in krajši opisi, ampak večinoma pa se ta korak preskoči in je večji poudarek na prototipu.
	\item \textbf{Prototip}. Je najpomembnejši korak v procesu razvoja igre. Sama definicija pomeni razvoj osnovnega produkta, ki je ustvarjen z namenom testiranja koncepta \cite{blackwell2015prototype}. Glavni namen je preizkusit, ali je glavna ideja igre dovolj dobra, da iz nje nastane končni produkt. Prototipiranje je zelo povezano z samim koncipiranjem in projektnim dokumentom. Tukaj se identificirajo pomanjkljivosti, ki jih je potrebno razčistiti in mogoče znova koncipirati, ter prednosti, ki jih je potrebno poudariti v končnem produktu. Dober prototip je odličen mejnik v razvoju, ki se uporablja kot izhodiščna točka za začetek pravega projekta. Obenem služi kot prikaz koncepta možnim investitorjem, ki bi hoteli vložiti v projekt. 
	\item \textbf{Produkcija}. Je glavni del razvoja projekta. Sodelujejo vsi člani ekipe opisani v poglavju \ref{sec:opis_problema}. Konstantna komunikacija med razvijalci, preizkuševalci in oblikovalci pelje projekt skozi faze razvoja programske opreme: \textit{pre-alpha, alpha, beta, release candidate, live release (gold)}. Odvisno od projekta, se lahko studio odloči, da izdajo produkt končnim uporabnikom še v fazi razvoja. Tako postanejo uporabniki del preizkuševalcev, ki pomagajo pri razvoju projekta.
	\item \textbf{Izid}. Pomeni konec glavnega procesa razvoja. Igra se zapakira in distribuira do končnih uporabnik preko posrednikov. Ker mine kar nekaj časa od verzije, ki jo razvijalci dajo distribuiran na fizičnih medijih, do dejanskega izida, je danes ustaljena praksa prenos popravkov preko spleta na dan izida. Nekoč se je tukaj razvoj igre zaključil, razvijalci pa so prehajali na naslednji projekt. Dandanes pa izid pomeni le še en mejnik v samem življenjskem ciklu igre.
	\item \textbf{Podpora po izidu} . Tako imenovan \textit{post production} v angleščini postaja danes vedno večji faktor. Razvoj igre je veliki finančni podvig, ki največ sredstev terja do izida igre. Zato se veliki studii raje posvetijo izdani igri z nadaljnjo podporo. Začel se je uporabljati termin igre kot storitev (angl. \textit{games as a service}). Studii podpirajo igro še vrsto let z novimi vsebinami, popravki ter včasih korenitimi spremembami s ciljem maksimirati dobiček. Konstantna komunikacija z ciljno publiko je pomembna, saj tako identificirajo, kaj si igralci želijo novega v igri.
\end{enumerate}
Tukaj ena lepa slikica procesa. $:)$

\cleardoublepage
\bibliographystyle{plain}
\bibliography{thesis}
\addcontentsline{toc}{chapter}{Literatura}
\thispagestyle{fancy}

\end{document}













